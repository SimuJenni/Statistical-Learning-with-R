\documentclass{paper}

%\usepackage{times}
\usepackage{epsfig}
\usepackage{graphicx}
\usepackage{amsmath}
\usepackage{amssymb}
\usepackage{color}
\usepackage{caption}
\usepackage{subcaption}


% load package with ``framed'' and ``numbered'' option.
%\usepackage[framed,numbered,autolinebreaks,useliterate]{mcode}

% something NOT relevant to the usage of the package.
\setlength{\parindent}{0pt}
\setlength{\parskip}{18pt}
\graphicspath{{images/}}


\usepackage[latin1]{inputenc}
\usepackage[T1]{fontenc}


\usepackage{listings}
\lstset{%
   language=R,
   basicstyle=\small\ttfamily,
   frame=single
}



\title{Assignment 9}



\author{Jenni Simon\\09-116-005}
% //////////////////////////////////////////////////


\begin{document}



\maketitle


% Add figures:
%\begin{figure}[t]
%%\begin{center}
%\quad\quad   \includegraphics[width=1\linewidth]{ass2}
%%\end{center}
%
%\label{fig:performance}
%\end{figure}


\paragraph{Exercise 1}

I performed logistic regression using all the predictors (no model-selection performed). The resulting model achieved a training accuracy of 84.8\%. The associated confusion-matrix is shown in Table \ref{tab:log}. 

\begin{table}[!h]
\centering
\caption{Confusion matrix for Logistic Regression on the training data}
\begin{tabular}{|c|c|c|}
\hline
 & Predicted: Abnormal & Predicted: Normal \\ \hline
Actual: Abnormal          & 186         &       23                              \\ 
Actual: Normal              & 24        &         77                            \\ \hline
\end{tabular}
\label{tab:log}
\end{table}


\paragraph{Exercise 2}

I performed linear discriminant analysis using all the predictors (no model-selection performed). The resulting model achieved a training accuracy of 85.8\%. The associated confusion-matrix is shown in Table \ref{tab:log}. 

\begin{table}[!h]
\centering
\caption{Confusion matrix for Linear Discriminant Analysis on the training data}
\begin{tabular}{|c|c|c|}
\hline
 & Predicted: Abnormal & Predicted: Normal \\ \hline
Actual: Abnormal          & 193        &     27                                \\ 
Actual: Normal              & 17         &       73                              \\ \hline
\end{tabular}
\label{tab:lda}
\end{table}


\paragraph{Exercise 3}

To compare the models I performed 10-fold cross validation with identical folds and used the resulting per fold error-rates to perform a paired t-test. The results of the t-test are shown in Listing \ref{list:res}. No significant difference in the mean error-rate could be detected, suggesting that the models are comparable in performance. The mean error-rate of the logistic regression model was 15.5\% and the mean error-rate achieved with LDA was 16.1\%. I would therefor favour the logistic regression model due to the lower mean error-rate in cross-validation.

\begin{minipage}{\linewidth}
  \begin{lstlisting}[caption={Summary of the logistic regression model.},
    label=list:res]
Paired t-test

data:  glm.errors and lda.errors
t = -0.32733, df = 9, p-value = 0.7509
alternative hypothesis: difference in means is not equal to 0
95 percent confidence interval:
 -0.05103874  0.03813551
sample estimates:
mean of the differences 
           -0.006451613 
  \end{lstlisting}
\end{minipage}







\end{document}
