\documentclass{paper}

%\usepackage{times}
\usepackage{epsfig}
\usepackage{graphicx}
\usepackage{amsmath}
\usepackage{amssymb}
\usepackage{color}
\usepackage{caption}
\usepackage{subcaption}


% load package with ``framed'' and ``numbered'' option.
%\usepackage[framed,numbered,autolinebreaks,useliterate]{mcode}

% something NOT relevant to the usage of the package.
\setlength{\parindent}{0pt}
\setlength{\parskip}{18pt}
\graphicspath{{images/}}


\usepackage[latin1]{inputenc} 
\usepackage[T1]{fontenc} 


\usepackage{listings} 
\lstset{% 
   language=Matlab, 
   basicstyle=\small\ttfamily, 
} 



\title{Assignment 3}



\author{Jenni Simon\\09-116-005}
% //////////////////////////////////////////////////


\begin{document}



\maketitle


% Add figures:
%\begin{figure}[t]
%%\begin{center}
%\quad\quad   \includegraphics[width=1\linewidth]{ass2}
%%\end{center}
%
%\label{fig:performance}
%\end{figure}



\paragraph{Exercise 1}
Please refer to the code for the implementation...

\paragraph{Exercise 2}
Please refer to the code for the implementation...

\paragraph{Exercise 3}
The results of the tests are summarised in table \ref{tab:res1}.
We observe that only 3 samples result in a p-value smaller than $0.05$. That is, in seven of the 20 samples we would not reject $H_0: \mu=1.5$ and therefore the test does only return the correct answer in seven of the 20 cases.


The results of the final t-test with all the 400 random values are shown in table \ref{tab:res2}. We observe a very small p-value in this case and are confident in rejecting the null-hypothesis in this case.

% latex table generated in R 3.2.3 by xtable 1.8-2 package
% Mon Mar 14 18:45:09 2016
\begin{table}[!h]
\centering
\begin{tabular}{lrrrr}
  \hline
 & Estimated Mean & Lower Bound & Upper Bound & p-Value \\ 
  \hline
1 & 2.3351320 & 1.5594636 & 3.1108005 & 0.0362362 \\ 
  2 & 1.9266657 & 1.3544401 & 2.4988912 & 0.1351176 \\ 
  3 & 1.7549305 & 1.1746239 & 2.3352371 & 0.3693754 \\ 
  4 & 2.1664972 & 1.2977806 & 3.0352139 & 0.1248079 \\ 
  5 & 2.0282766 & 1.3779259 & 2.6786273 & 0.1054140 \\ 
  6 & 1.9164520 & 1.2814165 & 2.5514876 & 0.1858660 \\ 
  7 & 1.9128686 & 1.4469787 & 2.3787585 & 0.0792072 \\ 
  8 & 1.8441143 & 1.1578699 & 2.5303586 & 0.3070979 \\ 
  9 & 1.3951930 & 0.6714262 & 2.1189598 & 0.7651179 \\ 
  10 & 2.2350091 & 1.7097991 & 2.7602191 & 0.0086095 \\ 
  11 & 2.2184988 & 1.5349824 & 2.9020152 & 0.0403693 \\ 
  12 & 2.2720401 & 1.5953447 & 2.9487356 & 0.0274862 \\ 
  13 & 1.9426886 & 1.2160205 & 2.6693568 & 0.2176538 \\ 
  14 & 2.4721206 & 1.8610119 & 3.0832293 & 0.0035233 \\ 
  15 & 1.9028432 & 1.2747726 & 2.5309139 & 0.1952675 \\ 
  16 & 2.5228839 & 2.0024516 & 3.0433162 & 0.0005909 \\ 
  17 & 1.8695263 & 1.1689361 & 2.5701164 & 0.2833938 \\ 
  18 & 2.3905279 & 1.7729180 & 3.0081378 & 0.0070752 \\ 
  19 & 1.6815761 & 1.1193657 & 2.2437865 & 0.5071986 \\ 
  20 & 1.7849574 & 1.1942924 & 2.3756224 & 0.3253033 \\ 
   \hline
\end{tabular}
\caption{Results of the 20 t-tests on samples of size 20 each. Shown are the computed sample mean, the bounds of the 95\% confidence interval and the p-value.}
\label{tab:res1}
\end{table}

\begin{table}[!h]
\centering
\begin{tabular}{rrrr}
  \hline
 Estimated Mean & Lower Bound & Upper Bound & p-Value \\ 
  \hline
2.02864 & 1.893207 & 2.164073 & 1.296e-13 \\ 
   \hline
\end{tabular}
\caption{Results of the final t-tests with all of the 400 random values. Shown are the computed sample mean, the bounds of the 95\% confidence interval and the p-value.}
\label{tab:res2}
\end{table}


 \end{document}